
\title{Résumé du cas d'utilisation --- Exécuter un programme}
        \subparagraph{Acteurs}
        Programmeur : Il lance l'exécution du programme présentement chargé dans l'interpréteur.

        \subparagraph{Objectifs}
        Le but est d'exécuter les instructions du programme chargé.

        \subparagraph{Pré-conditions, Post-conditions}

            \subparagraph{Pré-conditions}
            Toutes les instructions chargées sont correctes.

            \subparagraph{Post-conditions}
            Le contexte (variables) de l'interpréteur inclus le contexte final du programme.

        \subparagraph{Scénario nominal (grandes étapes)}
            \begin{enumerate}
            	\item Le programmeur exécute la commande \textbf{lance}.
            	\item L'interpréteur exécute l'instruction ayant l'étiquette la plus petite.
            	\item L'interpréteur passe l'instruction suivante (étiquette supérieure la plus proche sauf si changement du compteur ordinal).
            	\item Tant qu'il reste des instructions avec une étiquette supérieure retour en 3.
            	\item Le programme a fini de s'exécuter.
            	\item Le contrôle est rendu au programmeur qui peut à nouveau saisir.
            \end{enumerate}

        \subparagraph{Scénarios d'échec}
            \textbf{Point 2 du scénario nominal :} Aucune instruction est chargée dans l'interpréteur
            \par - L'interpréteur affiche un message d'erreur explicite.
            \par - Retour au point 6 du scénario nominal.
