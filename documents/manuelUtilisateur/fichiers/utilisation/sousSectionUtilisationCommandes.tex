\begin{description}
	\item[\textbf{Debut :}] Sur un programme chargé en mémoire centrale j'efface toutes les lignes de code et variables déclarées avec la commande \verb|debut|.
	
	\item[\textbf{Fin :}] Je suis en train d'utiliser l'interpréteur, je quitte l'interpréteur pour la session courante en exécutant la commande \verb|fin|.
	
	\item[\textbf{Defs :}] Les variables sont définies dans la session courante de l'interpréteur j'affiche le contexte actuel en exécutant la commande \verb|defs|.
	
	\item[\textbf{Lance :}] A partir de lignes d'instructions chargées dans la session courante de l'interpréteur LIR lorsque j'entre la commande \verb|lance| sans arguments et la valide, le programme s'exécute à partir de l'étiquette la plus petite.\\
	 Et avec argument les lignes d'instructions chargées dans la session courante de l'interpréteur LIR et lorsque j'entre la commande \verb|lance| sans arguments et la valide le programme s'exécute à partir de l'étiquette passé en argument.\\
	
	Exemple : \verb|lance| <étiquette début> : <étiquette fin>
	
	\item[\textbf{Efface :}] A partir d'une ou plusieurs lignes de programme mémorisées et leur étiquettes\\ on tape la commande: \verb|efface| <étiquette début> : <étiquette fin>.\\
	L'interpréteur effacera alors les lignes de programme dont le numéro d'étiquette est compris dans la plage.
	
	\item[\textbf{Liste :}]  J'entre la commande \verb| liste <etiquette_debut>:<etiquette_fin> | l'interpréteur va afficher toutes les lignes de programme mémorisées, dans l'ordre croissant de leur étiquette et dont les étiquettes sont situées dans cet intervalle donné. Et sans argument la commande \verb| liste | affiche toutes les lignes de programme mémorisées, dans l'ordre croissant de leur étiquette.
	
	\item[\textbf{Sauve :}] Quand programme (avec des étiquettes) ai été saisi, lorsque on entre la commande \verb|sauve| avec en argument le chemin du fichier (dans lequel on souhaite sauvegarder le travail),
	\verb|sauve <cheminFichier>| les lignes de codes tapées dans l'interpréteur s'enregistres dans le fichier passé en argument de la commande
	pour pouvoir être rechargées plus tard par l'interpréteur LIR avec la commande \verb|charge <cheminFichier>|.
	
	
	
	\item[\textbf{Charge :}] On a un fichier contenant un programme LIR dans notre ordinateur lorsque j'entre la commande \verb|charge| avec en argument le chemin de ce fichier les lignes de codes enregistrées dans le fichier sont chargée dans le
	programme pour pouvoir être exécutées et/ou modifiées par l'interpréteur LIR.
	
	Exemple : \verb|sauve <D:\leCheminDuFichier\leFichier.LextensionDuFichier>| OU \verb|sauve <leFichier.LextensionDuFichier>|
	
\end{description} 