Exemples d'utilisation de l'interpréteur LIR :
\begin{verbatim}
	TODO
\end{verbatim}

Exemples de programmes en LIR :
\begin{enumerate}
	\item \textbf{Exemple d'un programme demandant un état civil simple : }
\begin{verbatim}
10 affiche "Entre ton nom : " 
20 entre $nom 
30 affiche "Bienvenue "+$nom 
35 affiche 
40 var an=2021 
50 affiche "Quelle est ton année de naissance ? " 
60 entre naissance 
65 si naissance > an vaen 50 
70 affiche "Tu as autour de " 
80 affiche an-naissance 
90 affiche "ans " 
100 affiche 
200 stop
\end{verbatim}
    \underline{Exemple d'exécution :}
\begin{verbatim}
? charge etatCivilSimple.lir
ok
? lance
Entre ton nom : Emmanuel MACRON
Bienvenue Emmanuel MACRON
Quelle est ton année de naissance ? 1977
Tu as autour de 44ans 
\end{verbatim}
    \item \textbf{Exemple d'un programme calculant la factorielle à partir d'un entier saisi par l'utilisateur :}
\begin{verbatim}
10 affiche "Bienvenue dans le programme factorielle.lir !"
20 affiche
30 affiche "Entrez un entier : "
40 entre entier
45 procedure 500
50 procedure 1000
60 affiche entier
70 affiche "! = "
80 affiche factorielle
90 affiche
200 stop
    		
500 si entier >= 0 vaen 600
510 affiche "n! est définie sur l'ensemble des entiers naturels"
520 stop
600 retour 

1000 var factorielle = 1
1010 var entierCourant = 2
1011 var ancienFactorielle = factorielle
1012 var test = factorielle
1015 si entierCourant > entier vaen 1100
1016 si ancienFactorielle <> test vaen 1060
1017 var ancienFactorielle = factorielle
1020 var factorielle = factorielle * entierCourant
1025 var test = factorielle / entierCourant
1030 var entierCourant = entierCourant + 1
1040 vaen 1015
1050 vaen 1100
1060 affiche "dépassement de la capacité des entiers pour "
1070 affiche entier
1080 affiche "!"
1090 affiche
1095 stop
1100 retour
\end{verbatim}
    \underline{Exemple d'exécution :}
\begin{verbatim}
? charge factorielle.lir
ok
? lance
Bienvenue dans le programme factorielle.lir !
Entrez un entier : 10
10! = 3628800
\end{verbatim}
    \item \textbf{Exemple d'un programme déterminant l'entier médian de 3 entiers saisis :}
\begin{verbatim}
10 affiche "Bienvenue dans le programme Median3Entiers.lir"
20 affiche
30 affiche "Entrez le premier entier : "
40 entre premier
50 affiche "Entrez le deuxième entier : "
60 entre deuxieme
70 affiche "Entrez le troisième entier : "
80 entre troisieme
90 procedure 1000
100 affiche "Median( "
110 affiche premier
120 affiche ", "
130 affiche deuxieme
140 affiche ", "
150 affiche troisieme
160 affiche ") = "
170 affiche median
180 affiche
250 stop
        
1000 si premier <= deuxieme vaen 1100
1010 si deuxieme <= troisieme vaen 1200
1020 vaen 1520
        
1100 si deuxieme <= troisieme vaen 1520
1110 si premier <= troisieme vaen 1540
1120 vaen 1500
        
1200 si premier <= troisieme vaen 1500
1220 vaen 1540

1500 var median = premier
1510 vaen 1550
1520 var median = deuxieme
1530 vaen 1550
1540 var median = troisieme
1550 retour
\end{verbatim}
    \underline{Exemple d'exécution :}
\begin{verbatim}
? charge Median3Entiers.lir
ok
? lance
Bienvenue dans le programme Median3Entiers.lir
Entrez le premier entier : 55
Entrez le deuxième entier : 27
Entrez le troisième entier : 96
Median( 55, 27, 96) = 55
\end{verbatim}
\end{enumerate}

