\documentclass[12pt,a4paper]{article}
\usepackage[utf8]{inputenc}
\usepackage[T1]{fontenc}
\usepackage[french]{babel}
\title{Résumé de cas d'utilisation --- Sauvegarder le programme dans un fichier}
\date{} % laisser vide
\author{} % Laisser vide
    \begin{document}

        \maketitle

        \section{Acteurs}
        Programmeur : il entre la commande de sauvegarde `sauve' suivit du chemin vers le nom du fichie dans lequel on veut sauvegarder le programme.

        \section{Objectifs}
        L'objectif est de sauvegarder le programme rédigé dans l'interpréteur, dans un fichier texte 

        \section{Pré-conditions, Post-conditions}

            \subsection{Pré-Condtions}
            \begin{itemize}
            	\item Un programme doit être rédigé (au moins une ligne)
            	\item Le chemin vers le fichier ne doit pas contenir de caractères spéciaux(pour éviter les erreur)
        	\end{itemize}

            \subsection{Post-Conditions}
            \begin{itemize}
            	\item Le fichier doit être crée (si il n'existe pas déjà)
            	\item Le fichier doit contenir le programme rédigé par le programmeur
            \end{itemize}

        \section{Scénario nominal (grandes étapes)}
        \begin{enumerate}
        	\item Le programmeur execute la commande de sauvegarde
        	\item Le programme empèche la saisie à l'utilisateur
        	\item Le programme sauvegarde le code saisi (en mettant les étiquettes dans l'ordre croissant) et l'enregistre dans un fichier
        	\item Le programme affiche un message qui indique la fin de la sauvegarde
        	\item Le programme permet la saisie à l'utilisateur
    	\end{enumerate}

        \section{Scénarios d'échec}
        \begin{itemize}
        	\item Point 2 : Si le chemin du fichier éxécuté dans la commande de sauvegarde contiens des caractères spéciaux;
        	\begin{enumerate}
        		\item Affiche un message d'erreur spécifiant qu'il ne faut pas de caractères spéciaux dans le chemin
        		\item Reprend au point 5
        	\end{enumerate}

        	\item Point 3 : Si aucun programme n'as été écrit;
        	\begin{enumerate}
        		\item Affiche un message d'erreur spécifiant qu'il faut déjà avoir rédigé le programme avant de le sauvegarder
        		\item Reprend au point 5
        	\end{enumerate}

           \item Point 3 : Si aucun programme comporte plus de 99999 lignes;
        	\begin{enumerate}
        		\item Affiche un message d'erreur spécifiant que le nombre de lignes dépasse la valeur maximale
        		\item Reprend au point 5
        	\end{enumerate}
    	\end{itemize}

    \end{document}