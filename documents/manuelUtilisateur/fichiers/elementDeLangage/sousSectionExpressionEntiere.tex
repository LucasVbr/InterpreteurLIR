Les expressions concernent toujours deux opérandes séparés par un opérateur (notation infixe).
La présence d’espace (séparateur neutre) n’est pas obligatoire (mais conseillée).
Un opérande est soit une constante littérale, soit un identificateur.
L’opérateur est un caractère symbolisant les opérations arithmétiques courantes : +, -, *, /, \% \\

\begin{description}
	\item \verb|+|: addition entière
	\item \verb|-|: soustraction entière
	\item \verb|*|: multiplication entière
	\item \verb|/|: quotient de la division entière
	\item \verb|%|: reste de la division entière 
\end{description}