% \chapter{Tests du paquetage interpreteurlir.donnees.litteraux}

\section{Litteral}

Lors des itérations 1, la classe Litteral a été développée comme une classe
non-abstract, en effet nous n'avions pas encore abordé cette notion en cours.
Ainsi cette classe a été développée en TDD et nous avons par conséquent
effectué les tests unitaires de cette classe.
\\ Cependant à la fin de l'itération 3, dans une optique d'amélioration des
codes sources, nous avons passé cette classe en abstract, ainsi les tests
unitaires menés n'avaient plus lieu d'être et ont donc été tout de même
conservés en commentaire.

\section{Chaine}

Les jeux de tests de la classe Chaine prennent en compte les cas de chaînes
vide, la taille maximale des chaînes, leur syntaxe (avec le contenu de la
chaîne entre "). Aussi l'opération de concaténation a été testée.
Tous les tests menés ont été concluants.

\section{Entier}

La classe Entier est très proche de la classe Integer existant déjà dans le
JDK, ainsi son développement a été rapide.
À l'instar des tests menés pour la classe Chaine, tous les tests de la classe
Entier notamment des opérations arithmétiques ont été concluants.
\\ Notons, que pour les opérations arithmétiques telles que la division et le
reste de la division, le cas particulier de la division par zéro a été testé
à part.

\section{Booleen}

La classe Booleen n'a posé aucun problème particulier.
