    \section{Expression arithmétique}
    \subsection*{Récit d'utilisation}

    \paragraph{Titre : } Expression arithmétique
    \paragraph{Récit : } Calcul à l'aide d'expression arithmétique
    \paragraph{En tant que : } Programmeur
    \paragraph{Je souhaite : } que l'Interpréteur LIR effectue une
    opération arithmétique courante (addition, soustraction,
    multiplication, quotient ou reste d'une division entière)
    \paragraph{Afin que : } j'en exploite ou vois le résultat

    \subsection*{Critères d'acceptation}

    \paragraph{À partir de : } d'une ligne de l'interpréteur ou d'une
    ligne de programme à mémoriser et d'identificateurs auxquels une
    valeur aura été affectée préalablement ou de constantes littérales
    numérique.

    \paragraph{Alors : } j'entre une expression composée de deux
    opérandes de type entier signé et d'un opérateur.
    \\ Les opérandes peuvent être :
    \begin{itemize}
        \item deux constantes littérales
        \item deux identificateurs
        \item une constante littérale et un identificateur
    \end{itemize}
    \paragraph{Enfin : } j'obtiens le résultat de l'opération ou un
    message d'erreur m'informant que l'opération est impossible pour les
    identificateurs ou constantes littérales saisies.
