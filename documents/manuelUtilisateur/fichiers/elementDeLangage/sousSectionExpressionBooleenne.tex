Les expressions logiques ne sont utilisées qu'avec l'instruction si.
Les expressions logiques concernent donc toujours deux opérandes séparés par un opérateur relationnel (notation infixe).
Un opérande est soit une constante, soit un identificateur.
L’opérateur relationnel oprel est un symbole parmi : =, <>, <, <=, >, >= \\

\begin{description}
	\item \verb |=| : représente le test d’égalité
	\item \verb |<>| : représente l’inégalité
	\item \verb |<| : infériorité stricte
	\item \verb |>| : supériorité stricte
	\item \verb |<=| : inférieur ou égal
	\item \verb |=>| : supérieur ou égal
\end{description}

Ces opérateurs ont le sens habituel pour les entiers.
Pour les chaînes, c'est l'ordre lexicographique habituel qui sera utilisé. \\