\section{Commande sauve}

\subsection*{Récit d'utilisation}

\paragraph{Titre : } Commande sauve
\paragraph{Récit : } Sauvegarde d'un programme dans un fichier
\paragraph{En tant que : } Programmeur dans l'interpréteur LIR
\paragraph{Je souhaite : } sauvegarder un programme LIR dans un fichier
\paragraph{Afin de : } Pourvoir reprendre mon travail où je m'étais arrêté

\subsection*{Critères d'acceptation}

\paragraph{À partir du fait : } Qu'un programme (avec des étiquettes) ai été saisi
\paragraph{Alors : } lorsque j'entre la commande sauve avec en argument le chemin du fichier (dans lequel on souhaite sauvegarder le travail)
                     sauve <cheminFichier>
\paragraph{Enfin : } les lignes de codes tapées dans l'interpréteur s'enregistres dans le fichier passé en argument de la commande
                     pour pouvoir être rechargées plus tard par l'interpréteur LIR avec la commande charge <cheminFichier>
