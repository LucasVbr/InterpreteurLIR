\documentclass[12pt,a4paper]{article}
\usepackage[utf8]{inputenc}
\usepackage[T1]{fontenc}
\usepackage[french]{babel}
\title{Résumé de cas d'utilisation --- Charger un programme} % à remplacer
\date{} % laisser vide
\author{} % Laisser vide
\begin{document}
	
	\maketitle
	
	\section{Acteurs}
	Programmeur : Il entre la commande "charge" suivie du "chemin"/de l'arborescence du fichier que l'on veut charger.
	
	\section{Objectifs}
	Charger un programme en mémoire, en ayant pour seule indication son arborescence 
	
	\section{Pré-conditions, Post-conditions}
	Il faut que le programme ai été préalablement sauvegardé au stocké en mémoire
	
	\subsection{Pré-Condtions}
	L'interpréteur LIR est en mode édition. 
	Il faut que le programme ai été préalablement sauvegardé au stocké en mémoire, et qu'il soit un fichier texte.
	Et que sont chemin/arborescence soit accessible à l'interpréteur.
	
	\subsection{Post-Conditions}
	Le code source a été entièrement chargé sur LIR alors le chargement s'arrête.
	
	\section{Scénario nominal (grandes étapes)}
	\begin{enumerate}
		\item Le programmeur veut charger un fichier stocké.
		
		\item Le programmeur consulte l'arborescence de son fichier.
		
		\item Le programmeur entre la commande \verb|charge| suivie de l'arborescence de son fichier.
		
		\item L'interpréteur signale au programmeur que le chargement a pu se faire par un "ok". 
	\end{enumerate}
	
	\section{Scénarios d'échec}
	
	\paragraph{Point 2 du scénario nominal :} Aucun fichier n'est situé dans l'arborescence signalée 
	\begin{itemize}
		\item L'interpréteur en avise le programmeur au moyen d'un message d'erreur.
		\item Retour au point 1.
	\end{itemize}
	
	\paragraph{Point 3 du scénario nominal :} Le fichier ne correspond pas au type de fichier accepté par LIR.
	\begin{itemize}
		\item L'interpréteur affiche un message informant le programmeur.
		\item Retour au point 1.
	\end{itemize}
	
	\paragraph{Point 4 du scénario nominal :} La ligne de commande est incorrecte.
	\begin{itemize}
		\item Un message d'erreur en informe le programmeur
		\item Retour au point 1.
	\end{itemize}

	\paragraph{Point 4 du scénario nominal :} Le code source du fichier est corrompu
	\begin{itemize}
		\item Un message d'erreur en informe le programmeur
		\item Retour au point 1.
	\end{itemize}
	
	
\end{document}