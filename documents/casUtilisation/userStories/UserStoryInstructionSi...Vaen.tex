  
\documentclass[12pt,a5paper, notitle, oneside]{report}
\usepackage[utf8]{inputenc}
\usepackage[T1]{fontenc}
\usepackage[french]{babel}
\usepackage[landscape]{geometry}
\begin{document}
	
	\chapter*{Récit d'utilisation}
	
	\paragraph{Titre : } Instruction \verb|Si|...\verb|vaen|
	\paragraph{Récit : } Sauts conditionnels
	\paragraph{En tant que : } programmeur
	\paragraph{Je souhaite : } effectuer un saut vers une ligne
	spécifique d'un programme si la condition est remplie.
	\paragraph{Afin de : } Créer des branchements ou des itérations
	dans mes programmes.
	\newpage
	
	\chapter*{Critères d'acceptation}
	
	\paragraph{À partir de : } la saisie d'un programme
	\paragraph{Alors : } j'entre la commande \verb|si| suivie de la condition a remplir \verb|vaen| suivie du numéro
	de la ligne où je veux effectuer le saut.
	\paragraph{Enfin : } lors de l'exécution de l'instruction, le programme
	ignorera les lignes suivantes et sautera directement à la ligne
	indiquée si il valide la condition imposée.
	
\end{document}