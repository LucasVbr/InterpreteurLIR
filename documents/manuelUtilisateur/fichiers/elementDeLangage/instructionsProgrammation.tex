\begin{description}
	\item [\textbf{Var :}] Initialise une expression (chaine de caractère ou entier) dans variable définie par son identificateur, aussi utilisé pour changer sa valeur.

	\item [\textbf{Entre :}] Invite l'utilisateur a saisir une expression dans l'entrée standard. Celle-ci sera affectée dans la variable passée en argument.

	\item [\textbf{Affiche :}] Affiche le contenu de l'expression passé en argument sur la console de l'interpreteur. Si il n'y a pas d'argument cela provoque un saut de ligne.

	\item [\textbf{Vaen :}] Effectue un saut dans le programme vers l'étiquette passée en argument.

	\item [\textbf{Si ... vaen :}] Effectue un saut conditionnel, si l'expression booléenne passée en premier argument (entre si et vaen) est vrai, un saut s'effectue à l'étiquette passée en second argument (après vaen). 

	\item [\textbf{Procedure :}] Appel d'une fonction située à l'étiquette spécifiée. Puis reprend l'execution à la suite de cette action.

	\item [\textbf{Retour :}] Défini la fin d'une fonction, ordonne à l'interpreteur de retourner à la suite de l'instruction qui suit son appel.

	\item [\textbf{Stop :}] Arrête le programme.

\end{description}