\documentclass[12pt,a5paper, notitle, oneside]{report}
\usepackage[utf8]{inputenc}
\usepackage[T1]{fontenc}
\usepackage[french]{babel}
\usepackage[landscape, top=0.5cm]{geometry}
\begin{document}

    \chapter*{Récit d'utilisation}

    \paragraph{Titre : } Commande liste
    \paragraph{Récit : } Utilisation de la commande liste sans argument
    \paragraph{En tant que : } Programmeur
    \paragraph{Je souhaite : } que l'Interpréteur LIR affiche affiche
    toutes les lignes de programme mémorisées dans l'ordre
    croissant des numéros de ligne.
    \paragraph{Afin que : } je visualise ces lignes dans leur ordre
    d'exécution
    \newpage

    \chapter*{Critères d'acceptation}

    \paragraph{À partir de : } d'aucune ou plusieurs lignes de programme
    à mémoriser et de leurs étiquettes

    \paragraph{Alors : } j'entre la commande \verb| liste |

    \paragraph{Enfin : } l'interpréteur affiche toutes les lignes
    de programme mémorisées, s'il y en a, dans l'ordre croissant de leur
    étiquette.

\end{document}