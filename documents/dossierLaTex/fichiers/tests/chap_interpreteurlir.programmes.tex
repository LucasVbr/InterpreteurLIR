% \chapter{Tests du paquetage interpreteurlir.programmes}

\section{Etiquette}

Aucune difficulté n'a été rencontrée lors du développement de la classe
Etiquette, aussi les tests ont été concluants.

\section{Programme}

Lors de l'implémentation de la classe Programme, certaines méthodes ne pouvaient
être testées directement car il manquait encore des instructions permettant de le
faire.
Lors de l'implémentation de ces instructions, les tests de celles-ci ont permis
de tester également les méthodes de programmes. Ainsi certains tests de Programme
n'ont pu être menés que lors de l'intégration avec les instructions ou commandes.

\section{Les programmes de tests}

Lors de l'itération 2, alors que le commandes sauve et charge n'étaient pas encore
implémentées, un composant permettait de charger un programme complet au lancement
de l'interpréteur pour les démonstrations.
\\ Lors de l'itération 3, après l'implémentation de la commande charge, quatre
fichiers contenant un programme écrit en langage LIR ont été écrits pour les
démonstrations et tests finaux de l'interpréteur. Il s'agit de l'exemple de
programme proposé dans le cahier des charges et des programmes EtatCivil,
Median3Entiers et Factorielle.