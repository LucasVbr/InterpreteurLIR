    \section{Commande affiche avec une expression}

	\subsection*{Récit d'utilisation}

	\paragraph{Titre : } Commande affiche (expression)
	\paragraph{Récit : }  Afficher le contenu d'une expression sur la console de l'interpréteur
	\paragraph{En tant que : } Programmeur
	\paragraph{Je souhaite : } que l'interpréteur LIR évalue et affiche le contenu de l'expression que l'on lui donne
	\paragraph{Afin de : } pouvoir récupérer/vérifier le/les résultat(s) de son programme

	\subsection*{Critères d'acceptation}

	\paragraph{À partir du fait : } que j'ai la possibilité de saisir une ligne de commande
	\paragraph{Alors : } je tape la commande affiche et écrit l'expression dont je veut que la valeur soit affichée à la suite : affiche <expression>
	\paragraph{Enfin : } l'interpréteur évalue dans l'expression spécifiée la valeur de celle-ci et renvoie cette valeur sur la console et affiche un résultat sur LIR (en tant que feed-back) pour nous spécifier si la commande a bien pu s'exécuter
