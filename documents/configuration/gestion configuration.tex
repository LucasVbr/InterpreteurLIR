\documentclass[a4paper]{article}% autres choix : book, report
\usepackage[utf8]{inputenc}     % gestion des accents (source)
\usepackage[T1]{fontenc}        % gestion des accents (PDF)
\usepackage[french]{babel}    % gestion du français
\usepackage{textcomp}           % caractères additionnels
\usepackage{mathtools,amssymb,amsthm}% packages de l'AMS + mathtools
\usepackage{lmodern}            % police de caractère
\usepackage{geometry}           % gestion des marges
\usepackage{graphicx}           % gestion des images
\usepackage{xcolor}             % gestion des couleurs
\usepackage{array}              % gestion améliorée des tableaux
\usepackage{calc}               % syntaxe naurelle pour les calculs
\usepackage{titlesec}           % pour les sections
\usepackage{titletoc}           % pour la table des matières
\usepackage{fancyhdr}           % pour les en-têtes
\usepackage{titling}            % pour le titre
\usepackage{enumitem}           % pour les listes numérotées
\usepackage{hyperref}           % gestion des hyperliens
\hypersetup{pdfstartview=XYZ}   % zoom par défaut
\title{Gestion de la configuration}
\author{Nicolas Caminade \and Sylvan Courtiol \and Pierre Debas \and Heïa Dexter \and Lucas Vabre}
\date{17/04/2021}
\begin{document}
	\setlength{\parskip}{8pt}  % espacement après les paragraphes
	\maketitle
	\section*{Introduction}
	Ce document a pour but de confirmer par écrit la configuration logicielle choisie pour le projet.
	
	
	Le contenu de ce document n’est pas fixé et des changements peuvent être apportés. Cependant ce document doit être connu et suivi par les membres du groupe. En cas de modifications, une annonce sur discord sera faite.
	
	
	Pour toute question ou suggestion se référer au gestionnaire de configuration (présentement Sylvan COURTIOL).
	
	
    \section{Logiciels de développement}
        \subsection{Environnement de Développement Intégré}
        Eclipse JEE (version 2020-12)
        \par JDK 15
        
        \subsection{Contrôle des versions du code}
        Git (notamment intégré à Eclipse) avec dépôt sur GitHub. (Un apprentissage est nécessaire donc pour commencer certaines libertés sont possibles).
        
        \subsection{Organisation}
        Via le site Trello.
        
    \section{Logiciels généraux}
        \subsection{Communication}
        Les communications formelles sont effectuées via les mails de l’IUT (généralement par le chef de projet) avec les autres membres du projet en CC.
        
        
        Serveur discord spécifique au projet pour communication écrite ou vocale de la MOE.
        
        
        Google Meet pour les réunions avec les personnes autres que MOE. Adaptable à ce qui convient le mieux à cette personne.
        
        \subsection{Éditeur de Texte}
        LaTex sera mis en place au plus vite. (En attendant l’apprentissage par tous les membres du groupe de LaTex, les formats docx peuvent être utiliser).
        Les documents texte sont partagés soit en PDF soit en format modifiable par LaTex (ou docx au début)
        
        \subsection{Partage distant des fichiers}
        Les partages de tous les fichiers généraux et codes sources se feront sur GitHub via le site, git ou bien l’intégration Discord. (Un apprentissage est nécessaire donc pour commencer certaines libertés sont possibles).
        
    \section{Sécurité}
    Si possible tous les membres du groupe auront les mêmes droits sur les fichiers communs. En conséquence chaque membre du groupe ne doit pas donner des droits sur ces fichiers à une personne extérieure au projet (autre que MOA).
    
    
    Des sauvegardes du dépôt GitHub (contenant toutes les données du projets) seront effectuées régulièrement (fréquence à définir) par le gestionnaire de configuration. Toutes données qui ne sont pas dans le dépôt sont à la responsabilité de chacun.
    
\end{document}