\documentclass[12pt,a4paper]{article}
\usepackage[utf8]{inputenc}
\usepackage[T1]{fontenc}
\usepackage[french]{babel}

\title{Résumé de cas d'utilisation --- Exécuter une commande}
\date{} % laisser vide
\author{} % Laisser vide
\begin{document}

    \maketitle

    \section{Acteurs}
    Programmeur : il entre une commande à faire exécuter immédiatement par l'interpréteur.

    \section{Objectifs}
    Exécuter la commande entrée dans l'interpréteur.

    \section{Pré-conditions, Post-conditions}

    \subsection{Pré-Conditions}
    L'interpréteur LIR est lancé et le curseur est derrière l'invite.

    \subsection{Post-Conditions}
    La commande est exécutée et le résultat est affiché.

    \section{Scénario nominal (grandes étapes)}
        \begin{enumerate}
            \item Le programmeur écrit derrière l'invite une ligne de commande.
            \item Le programmeur valide cette commande.
            \item L'interpréteur effectue une analyse lexico-syntaxique.
            \item L'interpréteur interprète la ligne de commande.
        \end{enumerate}

    \section{Scénarios d'échec}
        \paragraph{Point 3 du scénario nominal :} la syntaxe de la ligne écrite est incorrecte.
        \begin{itemize}
            \item Un message d'erreur explicite informe le programmeur.
            \item Retour au point 4 du scénario nominal.
        \end{itemize}

        \paragraph{Point 4 du scénario nominal :} la commande conduit à une erreur d'exécution.
        \begin{itemize}
            \item Un message d'erreur explicite informe le programmeur.
            \item Retour au point 4 du scénario nominal.
        \end{itemize}

\end{document}