% \chapter{Tests du paquetage interpreteurlir.motscles}

\section{Commande}

Le développement de la classe Commande a été mené lors de l'itération 1
en TDD, en effet la classe a été passée en classe abstraite lors de
l'itération 3. Les tests mené ont été conservés en commentaire.

\section{EssaiCommande}

Lors de l'itération 1, avant l'implémentation de classe Analyseur, pour
l'intégration des premières commandes debut, defs et fin nous avons utilisé
une classe EssaiCommande. Une fois la classe Analyseur implémentée, l'intégration
des autres commandes a été par la suite testée avec.

\section{CommandeCharge}

Le développement de charge n'a pas été simple, en effet, son implémentation
a soulevé un problème de conception. La commande charge doit faire appel à
l'Analyseur cependant celui-ci n'a pas été conçu de façon assez générale pour
prendre en compte ce cas de figure. Au vu des délais à tenir, nous avons choisi
la solution qui nous paraissait la plus viable. Celle-ci impliquait de recréer
des parties de la classe Analyseur au sein même de la classe CommandeCharge.
\\ Les tests de charge ont encore une particularité, en effet, ils dépendent
de la machine sur laquelle les tests sont effectués, il faut donc adapter les
tests à la machine utilisée. Cela consiste à avoir sur la machine utilisée des
chemins d'accès et des fichiers coïncidant avec ceux des tests.

\section{CommandeDebut}
La commande debut a évolué entre l'itération 1 et 2 avec l'ajout de la remise
à zéro du programme en plus de celle du contexte.

\section{CommandeDefs et CommandeFin}

Le développement de ces commandes n'a pas posé de problème, aussi leurs
tests ont été concluants.

\section{CommandeEfface, CommandeLance et CommandeListe}

Le développement de ces trois commandes n'a pas posé de problème particulier.
Leurs tests ont permis de tester par la même occasion les méthodes de la classe
Programme.

\section{CommandeSauve}

Le développement de ces commandes n'a pas posé de problème.
À l'instar de charge, la commande sauve nécessite que les chemins pour les tests
soient accessibles sur la machine utilisée.