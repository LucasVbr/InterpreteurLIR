\documentclass[12pt,a5paper, notitle, oneside]{report}
\usepackage[utf8]{inputenc}
\usepackage[T1]{fontenc}
\usepackage[french]{babel}
\usepackage[landscape]{geometry}
\begin{document}
	
	\chapter*{Récit d'utilisation}
	
	\paragraph{Titre : } retour
	\paragraph{Récit : } Ordonner a l'interpréteur de retourner à la suite de l'instruction qui suit son appel.
	\paragraph{En tant que : } Programmeur
	\paragraph{Je souhaite : } retourner à la suite de la ligne de code qui a précédé l'appel de procédure.
	\paragraph{Afin de : } d'exécuter le programme qui allais s'exécuter si l'appel de procédure n'avait pas été fais.
	\newpage
	
	\chapter*{Critères d'acceptation}
	
	\paragraph{À partir de : } Plusieurs lignes de code et a la suite d'une instruction procédure.
	
	\paragraph{Alors : } En utilisant l'instruction \verb|retour|
	
	\paragraph{Enfin : } Alors l'interpréteur va chercher la ligne qui suivait l'instruction procédure et va l'exécuter jusqu'à'a la fin de la séquence.
	
\end{document}