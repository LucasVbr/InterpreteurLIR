    \section{Expression logique}

    \subsection*{Récit d'utilisation}

    \paragraph{Titre : } Expression logique dans un branchement
    conditionnel
    \paragraph{Récit : } Opérations relationnelles sur deux entiers
    \paragraph{En tant que : } Programmeur
    \paragraph{Je souhaite : } que l'Interpréteur LIR compare deux
    entiers avec une relation d'ordre ou d'équivalence
    \paragraph{Afin que : } d'exécuter ou non une branche du code avec
    l'instruction si

    \subsection*{Critères d'acceptation}

    \paragraph{À partir de : } d'une ligne de programme à mémoriser et d'identificateurs auxquels une valeur aura été affectée préalablement
    ou de constantes littérales de type entier signé.

    \paragraph{Alors : } j'entre une expression composée de deux
    opérandes de type entier signé et d'un opérateur et l'interpréteur
    évalue l'expression.
    \\ Les opérandes peuvent être :
    \begin{itemize}
        \item deux constantes littérales
        \item deux identificateurs
        \item une constante littérale et un identificateur
    \end{itemize}

    \paragraph{Enfin : } si l'expression (condition dans l'instruction)
    est vraie alors l'exécution continuera à partir du numéro de ligne
    spécifié par l’étiquette, sinon l'exécution continuera en séquence.