% \chapter{Organisation du projet}

    \section{Présentation du cycle de vie itératif} % TODO relecture
Pour développer l’Interpréteur LIR, le modèle de cycle de vie itératif a été
choisi. Ce modèle de développement de logiciel consiste en une succession de
cycles de spécification, de conception, de réalisation et de tests, le but
est d’enrichir et de « remodeler » des prototypes du logiciel successifs. Par
conséquent, une version du logiciel sera un « dernier prototype ».
\\La gestion du risque va entraîner la mise en place d’un noyau architectural
avec des fonctions indispensables du logiciel dès les deux premières
itérations. Les itérations suivantes apporteront des corrections et de
nouvelles fonctions au logiciel.
\\Les versions successives des prototypes permettent de matérialiser
l’avancement et d’éviter « l’effet tunnel » sur le projet. Ces prototypes
(versions 0.x) entretiennent la motivation des différents acteurs du projet :
l’équipe MOE, la MOA.
\\Le principe fondamental à chaque début d’itération est de ne spécifier en
détail que les fonctionnalités nécessaires pour cette itération. Ainsi la
prise en compte d’évolutions du besoin reste possible jusqu'à la dernière
itération. De même le « refactoring » de la conception (largement facilité
par les outils) a lieu à chaque étape pour intégrer des évolutions et des
ajouts. Le but étant bien sûr de fabriquer le logiciel adapté au besoin en
laissant la possibilité de « mûrir » au cours du temps.
\\Ce type de cycle implique une taille homogène de l’équipe et une
polyvalence des équipiers.

\section{Répartition des rôles}
Rôles des membres de l’équipe impliqués dans le projet jusqu'au mois de mai
2021 :
\begin{center}
    \begin{tabular}{rl}
        Chef de projet MOE            & Pierre Debas      \\
        Secrétaire de projet          & Heïa Dexter       \\
        Gestionnaire de configuration & Sylvan Courtiol   \\
        Développeur                   & Nicolas Caminade  \\
        Développeur                   & Lucàs Vabre       \\
    \end{tabular}
\end{center}

\section{Plan communication}
\subsection{Localisation géographique des intervenants}
L'équipe MOE, la MOA et les contrôleurs qualités sont basés sur Rodez (12).
\\La MOA, les contrôleurs qualités, H. Dexter sont basés sur Rodez (12), S.
Courtiol sur Luc-La-Primaube à côté de Rodez (12), P. Debas est basé à la
fois sur Rodez et à Albi (81), L. Vabre sur Gages (12) et N. Caminade sur
Rodez et Moncaut (47).

\subsection{Moyens de communication utilisés}
Les communications formelles sont effectuées via les mails de l’IUT
(généralement par le chef de projet) avec les autres membres du projet en CC.
\\Serveur Discord spécifique au projet pour communication écrite ou vocale de
la MOE.
\\Cf. le document Configuration interpréteur du langage LIR produit par le
gestionnaire de configuration (S. Courtiol).

\subsection{Réunions projets MOE}
Les réunions projet MOE seront hebdomadaires voire bi-hebdomadaires et dans
le contexte de la crise sanitaire elles se dérouleront en distanciel via
Discord (vocal, visio-conférence). Seront prévue des réunions courtes de 20
minutes et des réunions longues de 1h30.
\\Ces réunions auront pour objectif de faire le point sur l’avancement du
projet, le respect des objectifs fixé sur la période et de fixer les
prochains objectifs à remplir d’ici la prochaine réunions. Aussi ces réunions
seront l’occasion de faire part de difficultés éventuelles rencontrées par
les membres de l’équipe au cours de la semaine et de communiquer les
informations sur les prochaines rencontres avec la MOA.
\\Les comptes-rendus seront rédigés par la secrétaire de projet (H. Dexter)
et diffusés sur le serveur Discord de l’équipe sous format texte.

\subsection{Comités de Pilotage}
Les comités de pilotage rassembleront la MOA et toute l’équipe de MOE. Les
COPIL seront dirigé par le chef de projet éventuellement assisté par le
secrétaire.
\\La fréquence des COPIL est au mieux hebdomadaire et d’une durée d’une
demi-heure à trois quarts d’heure selon l’avancement du projet.
Les comptes-rendus des COPIL seront rédigés par l’actuelle secrétaire de
projet (H. Dexter) et diffusés le lendemain à la MOE du projet.


\section{Assurance qualité}
\subsection{Normes et standards de travail à observer (formalisme de modélisation, méthodes de contrôle, méthodes de développement, cycle de vie, conventions de code…)}
Pour mener à bien ce projet l'équipe MOE travaille
en utilisant le langage UML comme formalisme de modélisation, la méthode de
développement dirigé par les tests i.e. la méthode TDD (Test Driven
Development) en respectant les Java Code Convention pour un modèle de cycle
de vie itératif.

%\subsection{Manuel qualité et démarche qualité à observer (suivant la politique qualité de l’organisation), suivi et contrôle qualité (organisation, fréquence, participants).}


\section{Ressources matérielles et logicielles}
La partie qui suit est un résumé du document de gestion de la configuration joint
au dossier technique. Merci de vous y référer pour plus de détails.

        La conception en langage UML sera effectuée sous Modelio. La rédaction des
différents documents du dossier sera faite en utilisant \LaTeX. Nous utiliserons
Eclipse configuré avec un \emph{workspace} similaire à celui utilisé lors de nos
cours de programmation. Les dépôts en ligne et le contrôle de l'historique des
versions seront assurées par Git, plus précisément via GitHub. Lavancement sera
contrôlé sur un tableau Trello. Enfin, la communication sera assurée via un
serveur Discord dédié ou Google Meet pour les visio-conférences.