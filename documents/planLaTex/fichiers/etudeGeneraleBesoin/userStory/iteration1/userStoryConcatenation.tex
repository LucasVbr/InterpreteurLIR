    \section{Expression concaténation sur chaîne de caractères}

    \subsection*{Récit d'utilisation}

    \paragraph{Titre : } Opérateur + sur les chaînes de caractères
    \paragraph{Récit : } Concaténation de chaînes
    \paragraph{En tant que : } Programmeur
    \paragraph{Je souhaite : } accoler deux chaînes l'une à la suite de l'autre
    \paragraph{Afin de : } créer des messages dépendant du contexte d'exécution sur
    la sortie standard. Représenter une valeur entière par son écriture chiffrée en
    base 10.


    \subsection*{Critères d'acceptation}

    \paragraph{À partir de : } deux chaînes de caractères ou une chaîne et un entier,
    en tant qu'identificateurs déclarés ou expressions littérales.

    \paragraph{Alors : } En utilisant une expression de type
    \verb|var nouvelleChaine = opeGauche + opeDroite|, j'obtiens la concaténation de
    deux chaînes.

    \paragraph{Enfin : } L'identificateur \verb|nouvelleChaine| contient la chaîne
    constituée des deux primordiales concaténées. L'interpréteur confirme en affichant
    la nouvelle valeur ou m'informe d'une erreur. L'opération peut être récursive mais n'est pas commutative. Une concaténation s'effectue toujours par la droite.