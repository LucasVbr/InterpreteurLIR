    \subparagraph{Acteurs}
    Programmeur : il entre une commande à faire exécuter immédiatement par l'interpréteur.

    \subparagraph{Objectifs}
    Exécuter la commande entrée dans l'interpréteur.

    \subparagraph{Pré-Conditions}
    L'interpréteur LIR est lancé et le curseur est derrière l'invite.

    \subparagraph{Post-Conditions}
    La commande est exécutée et un résultat ou un feedback est affiché.

    \subparagraph{Scénario nominal (grandes étapes)}
        \begin{enumerate}
            \item Le programmeur écrit derrière l'invite une ligne de commande.
            \item Le programmeur valide cette commande.
            \item L'interpréteur effectue une analyse lexico-syntaxique.
            \item L'interpréteur interprète la ligne de commande.
        \end{enumerate}

    \subparagraph{Scénarios d'échec}
        \subparagraph{Point 3 du scénario nominal :} la syntaxe de la ligne écrite est incorrecte.
        \begin{itemize}
            \item Un message d'erreur explicite informe le programmeur.
            \item Retour au point 4 du scénario nominal.
        \end{itemize}

        \subparagraph{Point 4 du scénario nominal :} la commande conduit à une erreur d'exécution.
        \begin{itemize}
            \item Un message d'erreur explicite informe le programmeur.
            \item Retour au point 4 du scénario nominal.
        \end{itemize}

%\end{document}