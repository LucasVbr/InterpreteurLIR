    \chapter*{Introduction}
    \addcontentsline{toc}{chapter}{Introduction}
    \Large
    Le projet interpréteur LIR a été réalisé selon un modèle de cycle
    de vie itératif. Dans ce document des spécifications détaillées du
    projet seront présentés les fonctionnalités ajoutées à l'interpréteur
    au cours de chaque itération avec les récits d'utilisation des
    fonctionnalités ajoutées ou des éléments nécessaires au bon
    fonctionnement de l'interpréteur LIR.


    \large
    \chapter{Première itération}

    \section*{Contenu de la première itération}
    Dans cette première itération, l'objectif est d'avoir un
    prototype de l'interpréteur avec des fonctionnalités de base.
    Ces premières fonctionnalités sont les commandes et instructions suivantes :
    \begin{itemize}
        \item Commande \verb|debut| qui efface toutes les lignes de
              programme mémorisées ainsi que tous les identificateurs
              mémorisés.
        \item Commande \verb|fin| qui quitte l'interpréteur.
        \item Commande \verb|defs| qui affiche le contexte de
              l'interpréteur, i.e. affiche la liste des identificateurs
              définis durant la session avec leur valeur.
        \item Instruction \verb|affiche| qui évalue la valeur de
              l'expression et l'affiche sur la sortie texte courante ou
              alors provoque un saut de ligne sur la sortie texte
              courante.
        \item Instruction \verb|var| qui affecte la valeur de
              l'expression à la variable nommée par l’identificateur.
    \end{itemize}


    \footnotesize
    \chapter*{Récits d'utilisation proposés lors de l'itération 1}
        \section{Commande}

    \subsection*{Récit d'utilisation}

    \paragraph{Titre : } Exécution d'une commande
        \paragraph{Récit : } Exécution d'une commande
    \paragraph{En tant que : } programmeur avec l'interpréteur LIR
    \paragraph{Je souhaite : } exécuter une commande
    \paragraph{Afin de : } obtenir le résultat de cette commande ou une
                               confirmation de son exécution
    \subsection*{Critères d'acceptation}

    \paragraph{À partir du fait : } l'interpréteur affiche un invite
    \paragraph{Alors : } j'entre une ligne de commande
    \paragraph{Enfin : } j'obtiens le résultat de cette commande ou un retour
    m'informant du bon déroulé de l'exécution de la commande ou de son échec.
    \newpage
        \section{Commande debut}
    \subsection*{Récit d'utilisation}

    \paragraph{Titre : } debut
    \paragraph{Récit : } Réinitialiser l'environnement de l'interpréteur LIR
    \paragraph{En tant que : } programmeur
    \paragraph{Je souhaite : } vider l'intégralité du contexte d'exécution
    \paragraph{Afin de : } obtenir un environnement de travail vierge

    \subsection*{Critères d'acceptation}

    \paragraph{À partir de : } d'une session de l'interpréteur LIR
    \paragraph{Alors : } j'entre la commande \verb|debut|
    \paragraph{Enfin : } L'interpréteur efface toutes les lignes de programme
    mémorisées ainsi que tous les identificateurs mémorisés
        \section{Commande fin}

    \subsection*{Récit d'utilisation}

    \paragraph{Titre : } Quitter l'interpréteur (commande fin)
    \paragraph{Récit : } Quitter l'interpréteur (commande fin)
    \paragraph{En tant que : } programmeur avec l'interpréteur LIR
    \paragraph{Je souhaite : } quitter l'interpréteur
    \paragraph{Afin de : } arrêter d'utiliser l'interpréteur LIR pour la session courante

    \subsection*{Critères d'acceptation}

    \paragraph{À partir du fait : } je suis en train d'utiliser l'interpréteur
    \paragraph{Alors : } je souhaite quitter l'interpréteur pour la session courante en exécutant la commande fin
    \paragraph{Enfin : } le processus courant de l'interpréteur LIR s'arrête
        \section{Commande defs}
    \subsection*{Récit d'utilisation}

    \paragraph{Titre : } Affichages du contexte courant (commande defs) % Écrire le titre à la place du commentaire
    \paragraph{Récit : } Affichages du contexte courant (commande defs) % Écrire nom du récit à la suite
    \paragraph{En tant que : } programmeur avec l'interpréteur LIR % Remplacer commentaire par rôle
    \paragraph{Je souhaite : } voir toutes les variables définies dans la session courante (identificateur et valeur)
    \paragraph{Afin de : } connaître le contexte actuel de la session courante de l'interpréteur

    \subsection*{Critères d'acceptation}

    \paragraph{À partir du fait : } des variables sont définies dans la session courante de l'interpréteur
    \paragraph{Alors : } je souhaite connaître le contexte actuel en exécutant la commande defs
    \paragraph{Enfin : } l'interpréteur affiche chaque variable ligne par ligne avec son identificateur et sa valeur

        \section{Commande affiche}

	\subsection*{Récit d'utilisation}

	\paragraph{Titre : } Faire un saut de ligne avec la commande affiche
	\paragraph{Récit : }  Provoquer le saut de ligne sur la sortie de texte courante
	\paragraph{En tant que : } Programmeur
	\paragraph{Je souhaite : } que l'interpréteur LIR saute une ligne sur la sortie de texte courante
	\paragraph{Afin de : } Provoquer un saut de ligne sur cette sortie

	\subsection*{Critères d'acceptation}

	\paragraph{À partir du fait : } que j'ai une sortie de texte courante
	\paragraph{Alors : } j'entre la commande affiche
	\paragraph{Enfin : } l'interpréteur saute une ligne sur la sortie de texte courante
        \section{Commande affiche avec une expression}

	\subsection*{Récit d'utilisation}

	\paragraph{Titre : } Commande affiche (expression)
	\paragraph{Récit : }  Afficher le contenu d'une expression sur la console de l'interpréteur
	\paragraph{En tant que : } Programmeur
	\paragraph{Je souhaite : } que l'interpréteur LIR évalue et affiche le contenu de l'expression que l'on lui donne
	\paragraph{Afin de : } pouvoir récupérer/vérifier le/les résultat(s) de son programme

	\subsection*{Critères d'acceptation}

	\paragraph{À partir du fait : } que j'ai la possibilité de saisir une ligne de commande
	\paragraph{Alors : } je tape la commande affiche et écrit l'expression dont je veut que la valeur soit affichée à la suite : affiche <expression>
	\paragraph{Enfin : } l'interpréteur évalue dans l'expression spécifiée la valeur de celle-ci et renvoie cette valeur sur la console et affiche un résultat sur LIR (en tant que feed-back) pour nous spécifier si la commande a bien pu s'exécuter

        \section{Commande var pour une chaîne de caractères}

    \subsection*{Récit d'utilisation}

    \paragraph{Titre : } Commande var (Chaine de caractères)
    \paragraph{Récit : }  Initialiser une chaine de caractère dans variable / Changer sa valeur
    \paragraph{En tant que : } Programmeur
    \paragraph{Je souhaite : } que l'interpréteur LIR stock une chaine dans une variable
    \paragraph{Afin de : } pouvoir récupérer/manipuler cette chaine plus tard dans le programme


    \subsection*{Critères d'acceptation}

    \paragraph{À partir du fait : } que j'ai la possibilité de saisir une ligne de commande
    \paragraph{Alors : } je tape la commande var et met une chaine de caractère entre double guillements comme valeur : var <nomVariable>="<chaine>"
    \paragraph{Enfin : } l'interpréteur enregistre dans la variable spécifié la chaine de caractère voulue et renvoie la variable suivie de sa valeur (en tant que feed-back)
        \section{Commande var pour un entier}
   \subsection*{Récit d'utilisation}

    \paragraph{Titre : } Commande var (Entier)
    \paragraph{Récit : }  Initialiser un entier dans variable / Changer sa valeur
    \paragraph{En tant que : } Programmeur
    \paragraph{Je souhaite : } que l'interpréteur LIR stock un entier dans une variable
    \paragraph{Afin de : } pouvoir récupérer/manipuler cet entier plus tard dans le programme

    \subsection*{Critères d'acceptation}

    \paragraph{À partir du fait : } que j'ai la possibilité de saisir une ligne de commande
    \paragraph{Alors : } je tape la commande var et met un entier comme valeur :
    \verb |var <nomVariable>=<entier> |
    \paragraph{Enfin : } l'interpréteur enregistre dans la variable spécifié l'entier voulu et renvoie la variable suivie de sa valeur (en tant que feed-back)

        \section{Expression concaténation sur chaîne de caractères}

    \subsection*{Récit d'utilisation}

    \paragraph{Titre : } Opérateur + sur les chaînes de caractères
    \paragraph{Récit : } Concaténation de chaînes
    \paragraph{En tant que : } Programmeur
    \paragraph{Je souhaite : } accoler deux chaînes l'une à la suite de l'autre
    \paragraph{Afin de : } créer des messages dépendant du contexte d'exécution sur
    la sortie standard. Représenter une valeur entière par son écriture chiffrée en
    base 10.


    \subsection*{Critères d'acceptation}

    \paragraph{À partir de : } deux chaînes de caractères ou une chaîne et un entier,
    en tant qu'identificateurs déclarés ou expressions littérales.

    \paragraph{Alors : } En utilisant une expression de type
    \verb|var nouvelleChaine = opeGauche + opeDroite|, j'obtiens la concaténation de
    deux chaînes.

    \paragraph{Enfin : } L'identificateur \verb|nouvelleChaine| contient la chaîne
    constituée des deux primordiales concaténées. L'interpréteur confirme en affichant
    la nouvelle valeur ou m'informe d'une erreur. L'opération peut être récursive mais n'est pas commutative. Une concaténation s'effectue toujours par la droite.
        \section{Expression logique}

    \subsection*{Récit d'utilisation}

    \paragraph{Titre : } Expression logique dans un branchement
    conditionnel
    \paragraph{Récit : } Opérations relationnelles sur deux entiers
    \paragraph{En tant que : } Programmeur
    \paragraph{Je souhaite : } que l'Interpréteur LIR compare deux
    entiers avec une relation d'ordre ou d'équivalence
    \paragraph{Afin que : } d'exécuter ou non une branche du code avec
    l'instruction si

    \subsection*{Critères d'acceptation}

    \paragraph{À partir de : } d'une ligne de programme à mémoriser et d'identificateurs auxquels une valeur aura été affectée préalablement
    ou de constantes littérales de type entier signé.

    \paragraph{Alors : } j'entre une expression composée de deux
    opérandes de type entier signé et d'un opérateur et l'interpréteur
    évalue l'expression.
    \\ Les opérandes peuvent être :
    \begin{itemize}
        \item deux constantes littérales
        \item deux identificateurs
        \item une constante littérale et un identificateur
    \end{itemize}

    \paragraph{Enfin : } si l'expression (condition dans l'instruction)
    est vraie alors l'exécution continuera à partir du numéro de ligne
    spécifié par l’étiquette, sinon l'exécution continuera en séquence.
    \documentclass[12pt,a5paper, notitle, oneside]{report}
\usepackage[utf8]{inputenc}
\usepackage[T1]{fontenc}
\usepackage[french]{babel}
\usepackage[landscape, top=0.5cm]{geometry}
\begin{document}

    \chapter*{Récit d'utilisation}

    \paragraph{Titre : } Expression arithmétique
    \paragraph{Récit : } Calcul à l'aide d'expression arithmétique
    \paragraph{En tant que : } Programmeur
    \paragraph{Je souhaite : } que l'Interpréteur LIR effectue une
    opération arithmétique courante (addition, soustraction,
    multiplication, quotient ou reste d'une division entière)
    \paragraph{Afin que : } j'en exploite ou vois le résultat
    \newpage

    \chapter*{Critères d'acceptation}

    \paragraph{À partir de : } d'une ligne de l'interpréteur ou d'une
    ligne de programme à mémoriser et d'identificateurs auxquels une
    valeur aura été affectée préalablement ou de constantes littérales
    numérique.

    \paragraph{Alors : } j'entre une expression composée de deux
    opérandes de type entier signé et d'un opérateur.
    \\ Les opérandes peuvent être :
    \begin{itemize}
        \item deux constantes littérales
        \item deux identificateurs
        \item une constante littérale et un identificateur
    \end{itemize}
    \paragraph{Enfin : } j'obtiens le résultat de l'opération ou un
    message d'erreur m'informant que l'opération est impossible pour les
    identificateurs ou constantes littérales saisies.

\end{document}

    \large
    \chapter{Deuxième itération}
    \section*{Contenu de la deuxième itération}
    Dans cette deuxième itération, l'objectif est d'ajouter des
    fonctionnalités permettant l'écriture de programmes simple
    en LIR, à savoir :
    \begin{itemize}
        \item Commande \verb|efface| qui efface toutes les lignes
              de programme dont le numéro d’étiquette est dans la
              plage comprise entre \verb|<etiquette_debut>| et
              \verb|<etiquette_fin>|.
        \item Commande \verb|lance| qui démarre l’exécution d’un
              programme à partir de son plus petit numéro d’étiquette
              ou du numéro d'étiquette indiqué par l'utilisateur.
        \item Commande \verb|liste| qui affiche toutes les lignes de
              programme mémorisées dans l'ordre croissant des numéros
              de ligne.
        \item Instruction \verb|stop| qui arrête l'exécution du programme.
        \item Instruction \verb|vaen| qui continue l'exécution à partir
              du numéro spécifié par étiquette.
        \item Instruction \verb|procedure| qui transfère l'exécution du
              programme au numéro d’étiquette spécifié et qui reprendra
              en séquence lorsque la procédure sera terminée.
        \item Instruction \verb|retour| qui, rencontrée après un appel
              de procédure, provoque un retour à l'instruction qui suit
              son appel.
    \end{itemize}

    \footnotesize
    \chapter*{Récits d'utilisation proposés lors de l'itération 2}
    \section{Commande efface}

    \subsection*{Récit d'utilisation}

    \paragraph{Titre : } Commande efface
    \paragraph{Récit : } Utilisation de la commande efface
    \paragraph{En tant que : } Programmeur
    \paragraph{Je souhaite : } Supprimer ue ou plusieurs lignes d'un programme
    \paragraph{Afin de : } Effacer les instructions d'un bloc de code

    \subsection*{Critères d'acceptation}

    \paragraph{À partir de : } une ou plusieurs lignes de programme mémorisé et leur étiquettes
    \paragraph{Alors : } on tape la commande: efface <etiquette\_debut> : <etiquette\_fin>
    \paragraph{Enfin : } l'interpréteur efface les lignes de programme dont le numéro d'étiquette est compris dans la plage, comprise entre etiquette\_debut et etiquette\_fin

    \newpage
    \documentclass[12pt,a5paper, notitle, oneside]{report}
\usepackage[utf8]{inputenc}
\usepackage[T1]{fontenc}
\usepackage[french]{babel}
\usepackage[landscape]{geometry}
\begin{document}

    \chapter*{Récit d'utilisation}

    \paragraph{Titre : } Commande lance sans argument
    \paragraph{Récit : } Exécuter le programme à partir de l'étiquette la plus petite
    \paragraph{En tant que : } Programmeur avec l'interpréteur LIR
    \paragraph{Je souhaite : } Exécuter le programme chargé avec la commande lance
    \paragraph{Afin de : } obtenir le comportement du programme chargé pour atteindre son objectif
    \newpage

    \chapter*{Critères d'acceptation}

    \paragraph{À partir de : } lignes d'instructions chargé dans la session courante de l'interpréteur LIR
    \paragraph{Alors : } lorsque j'entre la commande lance sans arguments et la valide le programme s'exécute à 
                         partir de l'étiquette la plus petite
    \paragraph{Enfin : } le contexte de l'interpréteur contient le contexte final du programme exécuté

\end{document}
    \documentclass[12pt,a5paper, notitle, oneside]{report}
\usepackage[utf8]{inputenc}
\usepackage[T1]{fontenc}
\usepackage[french]{babel}
\usepackage[landscape]{geometry}
\begin{document}

    \chapter*{Récit d'utilisation}

    \paragraph{Titre : } Commande stop
    \paragraph{Récit : } Utilisation de la commande stop
    \paragraph{En tant que : } Programmeur
    \paragraph{Je souhaite : } Arreter un programme
    \paragraph{Afin de : } terminer son execution 
    \newpage

    \chapter*{Critères d'acceptation}

    \paragraph{À partir du fait : } Qu'un programme comporte au moins une instruction
    \paragraph{Alors : } on tape la commande: <etiquette> stop
    \paragraph{Enfin : } A son execution, le programme s'arrete lorsqu'il a atteind l'étiquette spécifiée.
			      Puis il rend la main à l'utilisateur qui pourra à nouveau tapper des commandes.

\end{document}
    \section{Etiquette}
    \subsection*{Récit d'utilisation}

    \paragraph{Titre : } Étiquettes
    \paragraph{Récit : } Ordonner les lignes d'un programme avec les étiquettes
    \paragraph{En tant que : } Programmeur avec l'interpréteur LIR
    \paragraph{Je souhaite : } ajouter des instruction au programmes dans un ordre précis
    \paragraph{Afin de : } que les instructions puissent être exécutées dans le bon ordre

    \subsection*{Critères d'acceptation}

    \paragraph{À partir de : } l'interpréteur LIR et des instructions définies
    \paragraph{Alors : } lorsque j'entre une instruction précédée d'une étiquette alors celle-ci est enregistrée avec son étiquette pour pouvoir être exécutée plus tard.
    \paragraph{Enfin : } lorsque le programme est lancé alors les instructions s'exécutent l'ordre des étiquettes.

    \documentclass[12pt,a5paper, notitle, oneside]{report}
\usepackage[utf8]{inputenc}
\usepackage[T1]{fontenc}
\usepackage[french]{babel}
\usepackage[landscape]{geometry}
\begin{document}

    \chapter*{Récit d'utilisation}

    \paragraph{Titre : } Instructions
    \paragraph{Récit : } Consulter et modifier le contexte d'exécution
    \paragraph{En tant que : } programmeur
    \paragraph{Je souhaite : } faire réaliser des actions par l'interpréteur
    \paragraph{Afin de : } déclarer des variables, des fonctions, effectuer des
    sauts conditionnels, des itérations, connaître et manipuler le contexte
    d'un programme.
    \newpage

    \chapter*{Critères d'acceptation}

    \paragraph{À partir de : } ligne de commande ou programme
    \paragraph{Alors : } J'entre une instruction pour effectuer une action précise
    \paragraph{Enfin : } Le contexte est modifié en fonction de cette instruction.
    L'interpréteur m'informe en cas d'erreur de syntaxe

\end{document}
    \documentclass[12pt,a5paper, notitle, oneside]{report}
\usepackage[utf8]{inputenc}
\usepackage[T1]{fontenc}
\usepackage[french]{babel}
\usepackage[landscape]{geometry}
\begin{document}

    \chapter*{Récit d'utilisation}

    \paragraph{Titre : } Instruction \verb|vaen|
    \paragraph{Récit : } Sauts inconditionnels
    \paragraph{En tant que : } programmeur
    \paragraph{Je souhaite : } effectuer un saut vers une ligne
        spécifique d'un programme.
    \paragraph{Afin de : } Créer des branchements ou des itérations
        dans mes programmes.
    \newpage

    \chapter*{Critères d'acceptation}

    \paragraph{À partir de : } la saisie d'un programme
    \paragraph{Alors : } j'entre la commande \verb|vaen| suivie du numéro
        de la ligne où je veux effectuer le saut.
    \paragraph{Enfin : } lors de l'exécution de l'instruction, le programme
        ignorera les lignes suivantes et sautera directement à la ligne
        indiquée.

\end{document}
    \section{Commande lance à partir d'une étiquette}

    \subsection*{Récit d'utilisation}

    \paragraph{Titre : } Commande lance <Étiquette>
    \paragraph{Récit : } Exécuter le programme à partir de l'étiquette argument
    \paragraph{En tant que : } Programmeur avec l'interpréteur LIR
    \paragraph{Je souhaite : } Exécuter le programme chargé avec la commande lance <étiquette>
    \paragraph{Afin de : } obtenir le comportement et objectif du programme chargé


    \subsection*{Critères d'acceptation}

    \paragraph{À partir de : } lignes d'instructions chargé dans la session courante de l'interpréteur LIR
    \paragraph{Alors : } lorsque j'entre la commande lance sans arguments et la valide le programme s'exécute à
                         partir de l'étiquette passé en argument
    \paragraph{Enfin : } le contexte de l'interpréteur contient le contexte final du programme exécuté à partir de l'étiquette spécifiée
    \section{Instruction procédure}

	\subsection*{Récit d'utilisation}

	\paragraph{Titre : } Procédure
	\paragraph{Récit : } Ordonner a l'interpréteur à exécuter des lignes de code à partir de l'étiquette de l'instruction.
	\paragraph{En tant que : } Programmeur
	\paragraph{Je souhaite : } transférer l'exécution au numéro d'étiquette spécifié.
	\paragraph{Afin de : } exécuter le programme puis reprendre en séquence une fois le procédure terminée.

	\subsection*{Critères d'acceptation}

	\paragraph{À partir de : } Plusieurs lignes de code et d'identificateurs déclarés, dont la portée est globale.

	\paragraph{Alors : } En utilisant l'instruction
    \verb|procedure <etiquette>|

	\paragraph{Enfin : } Alors l'interpréteur va chercher la ligne qui a pour identificateur celui référencé
						 en étiquette et va l'exécuter jusqu'à'a la fin de la séquence.
    \section{Instruction retour}

	\subsection*{Récit d'utilisation}

	\paragraph{Titre : } retour
	\paragraph{Récit : } Ordonner a l'interpréteur de retourner à la suite de l'instruction qui suit son appel.
	\paragraph{En tant que : } Programmeur
	\paragraph{Je souhaite : } retourner à la suite de la ligne de code qui a précédé l'appel de procédure.
	\paragraph{Afin de : } d'exécuter le programme qui allais s'exécuter si l'appel de procédure n'avait pas été fais.

	\subsection*{Critères d'acceptation}

	\paragraph{À partir de : } Plusieurs lignes de code et a la suite d'une instruction procédure.

	\paragraph{Alors : } En utilisant l'instruction \verb|retour|

	\paragraph{Enfin : } Alors l'interpréteur va chercher la ligne qui suivait l'instruction procédure et va l'exécuter jusqu'à'a la fin de la séquence.
    \section{Commande liste}

    \subsection*{Récit d'utilisation}

    \paragraph{Titre : } Commande liste
    \paragraph{Récit : } Utilisation de la commande liste avec argument
    \paragraph{En tant que : } Programmeur
    \paragraph{Je souhaite : } que l'Interpréteur LIR affiche affiche
    toutes les lignes de programme mémorisées dans l'ordre
    croissant des numéros de ligne dans un intervalle donné.
    \paragraph{Afin que : } je visualise uniquement les lignes de cet intervalle dans l'ordre croissant.

    \subsection*{Critères d'acceptation}

    \paragraph{À partir de : } aucune ou plusieurs lignes de programme
    à mémoriser et de leurs étiquettes et d'un intervalle d'entier passé en argument

    \paragraph{Alors : } j'entre la commande \verb| liste <etiquette_debut>:<etiquette_fin> |

    \paragraph{Enfin : } l'interpréteur affiche toutes les lignes
    de programme mémorisées, s'il y en a, dans l'ordre croissant de leur
    étiquette et dont les étiquettes sont situées dans cet intervalle donné.
    \documentclass[12pt,a5paper, notitle, oneside]{report}
\usepackage[utf8]{inputenc}
\usepackage[T1]{fontenc}
\usepackage[french]{babel}
\usepackage[landscape, top=0.5cm]{geometry}
\begin{document}

    \chapter*{Récit d'utilisation}

    \paragraph{Titre : } Commande liste
    \paragraph{Récit : } Utilisation de la commande liste sans argument
    \paragraph{En tant que : } Programmeur
    \paragraph{Je souhaite : } que l'Interpréteur LIR affiche affiche
    toutes les lignes de programme mémorisées dans l'ordre
    croissant des numéros de ligne.
    \paragraph{Afin que : } je visualise ces lignes dans leur ordre
    d'exécution
    \newpage

    \chapter*{Critères d'acceptation}

    \paragraph{À partir de : } d'aucune ou plusieurs lignes de programme
    à mémoriser et de leurs étiquettes

    \paragraph{Alors : } j'entre la commande \verb| liste |

    \paragraph{Enfin : } l'interpréteur affiche toutes les lignes
    de programme mémorisées, s'il y en a, dans l'ordre croissant de leur
    étiquette.

\end{document}

    \large
    \chapter{Troisième itération}
    \section*{Contenu de la première itération}
    Dans cette troisième itération, l'objectif est de couvrir toutes
    les fonctionnalités attendues. Ces dernières  concernent la
    lecture et l'écriture de fichier et l'ajout d'une structure de
    contrôle à l'interpréteur :
    \begin{itemize}
        \item Commande \verb|sauve| qui sauvegarde les lignes de
              programme dans le fichier texte indiqué en argument.
        \item Commande \verb|charge| qui charge dans le contexte
              les lignes de programme sauvegardées dans le fichier
              texte indiqué en argument.
        \item Instruction \verb|si... vaen| : si la condition est
              vraie alors l'exécution continuera à partir du numéro
              de ligne spécifié par l’étiquette, sinon l'exécution
              continuera en séquence.
    \end{itemize}

    \footnotesize
    \chapter*{Récits d'utilisation proposés lors de l'itération 3}
    \section{Commande sauve}

\subsection*{Récit d'utilisation}

\paragraph{Titre : } Commande sauve
\paragraph{Récit : } Sauvegarde d'un programme dans un fichier
\paragraph{En tant que : } Programmeur dans l'interpréteur LIR
\paragraph{Je souhaite : } sauvegarder un programme LIR dans un fichier
\paragraph{Afin de : } Pourvoir reprendre mon travail où je m'étais arrêté

\subsection*{Critères d'acceptation}

\paragraph{À partir du fait : } Qu'un programme (avec des étiquettes) ai été saisi
\paragraph{Alors : } lorsque j'entre la commande sauve avec en argument le chemin du fichier (dans lequel on souhaite sauvegarder le travail)
                     sauve <cheminFichier>
\paragraph{Enfin : } les lignes de codes tapées dans l'interpréteur s'enregistres dans le fichier passé en argument de la commande
                     pour pouvoir être rechargées plus tard par l'interpréteur LIR avec la commande charge <cheminFichier>

    \newpage
    \section{Commande charge}

\subsection*{Récit d'utilisation}

\paragraph{Titre : } Commande charge
\paragraph{Récit : } Chargement d'un programme à partir d'un fichier
\paragraph{En tant que : } Programmeur avec l'interpréteur LIR
\paragraph{Je souhaite : }  charger un programme LIR préalablement enregistré dans un fichier
\paragraph{Afin de : } je puisse réutiliser un programme LIR sans repartir de zéro.

\subsection*{Critères d'acceptation}

\paragraph{À partir du fait : } un fichier contenant un programme LIR sur mon ordinateur
\paragraph{Alors : } lorsque j'entre la commande charge avec en argument le chemin de ce fichier
\paragraph{Enfin : } les lignes de codes enregistrées dans le fichier sont chargée dans le
                     programme pour pouvoir être exécutées et/ou modifiées par l'interpréteur LIR

    \section{Instruction si... vaen}
	\subsection*{Récit d'utilisation}

	\paragraph{Titre : } Instruction \verb|Si|...\verb|vaen|
	\paragraph{Récit : } Sauts conditionnels
	\paragraph{En tant que : } programmeur
	\paragraph{Je souhaite : } effectuer un saut vers une ligne
	spécifique d'un programme si la condition est remplie.
	\paragraph{Afin de : } Créer des branchements ou des itérations
	dans mes programmes.
	\newpage
	\subsection*{Critères d'acceptation}

	\paragraph{À partir de : } la saisie d'un programme
	\paragraph{Alors : } j'entre la commande \verb|si| suivie de la condition a remplir \verb|vaen| suivie du numéro
	de la ligne où je veux effectuer le saut.
	\paragraph{Enfin : } lors de l'exécution de l'instruction, le programme
	ignorera les lignes suivantes et sautera directement à la ligne
	indiquée si il valide la condition imposée.
