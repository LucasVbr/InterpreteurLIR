    \section{Commande var pour une chaîne de caractères}

    \subsection*{Récit d'utilisation}

    \paragraph{Titre : } Commande var (Chaine de caractères)
    \paragraph{Récit : }  Initialiser une chaine de caractère dans variable / Changer sa valeur
    \paragraph{En tant que : } Programmeur
    \paragraph{Je souhaite : } que l'interpréteur LIR stock une chaine dans une variable
    \paragraph{Afin de : } pouvoir récupérer/manipuler cette chaine plus tard dans le programme


    \subsection*{Critères d'acceptation}

    \paragraph{À partir du fait : } que j'ai la possibilité de saisir une ligne de commande
    \paragraph{Alors : } je tape la commande var et met une chaine de caractère entre double guillements comme valeur : var <nomVariable>="<chaine>"
    \paragraph{Enfin : } l'interpréteur enregistre dans la variable spécifié la chaine de caractère voulue et renvoie la variable suivie de sa valeur (en tant que feed-back)